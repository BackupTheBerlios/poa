%%%%%%%%%%%%%%%%%%%%%%%%%%%%%%%%%%%%%%%%%%%%%%%%%%%%%%%%%%%%%%%%%%%%%%%%%%%%%%%
%% StuPro B, "Programmierumgebung Offener Antrieb" (POA)
%% Angebot
%% $Id: spezifikation.tex,v 1.2 2003/07/01 10:08:29 vanto Exp $
%%%%%%%%%%%%%%%%%%%%%%%%%%%%%%%%%%%%%%%%%%%%%%%%%%%%%%%%%%%%%%%%%%%%%%%%%%%%%%%
\documentclass[a4paper,titlepage,12pt,ngerman]{scrbook}
\usepackage{../common/header}

\RCSdef $Revision: 1.2 $
\RCSdef $Date: 2003/07/01 10:08:29 $

\newcommand\version{Version 1.0 \xspace}

\begin{document}

%%%%%%%%%%%%%%%%%%%%%%%%%%%%%%%%%%%%%%%%%%%%%%%%%%%%%%%%%%%%%%%%%%%%%%%%%%%%%%%
%% Deckblatt

\begin{titlepage}
\renewcommand{\thefootnote}{\fnsymbol{footnote}}
{\Huge
\raggedright
\textbf{POA} \\
\huge Programmierumgebung offener Antrieb
\rule{\textwidth}{0.75pt}
\par
}
\begin{flushleft}
\normalsize
\version
\end{flushleft}
\vfill

{\parindent=0cm
\Huge Spezifikation
}


\setcounter{footnote}{0}
\end{titlepage}

%%%%%%%%%%%%%%%%%%%%%%%%%%%%%%%%%%%%%%%%%%%%%%%%%%%%%%%%%%%%%%%%%%%%%%%%%%%%%%%
%% Versionsgeschichte

\section*{Versionsgeschichte}

\begin{itemize}

\item Version 1.0 (17.6.2003)

  Diese Version wurde dem Auftraggeber zur Abnahme vorgelegt.

\end{itemize}

%%%%%%%%%%%%%%%%%%%%%%%%%%%%%%%%%%%%%%%%%%%%%%%%%%%%%%%%%%%%%%%%%%%%%%%%%%%%%%%
%% Inhaltsverzeichnis

\tableofcontents

%%%%%%%%%%%%%%%%%%%%%%%%%%%%%%%%%%%%%%%%%%%%%%%%%%%%%%%%%%%%%%%%%%%%%%%%%%%%%%%
%% Einleitung

\chapter {Einleitung}
\section {Zielsetzung}
In diesem Dokument sind die Anforderungen an den Prototypen im Rahmen des Projekts POA (Programmierumgebung Offener Antrieb) 
detailiert spezifiziert. Die Entwicklung des Prototypen gilt als erfolgreich abgeschlossen, wenn das Produkt allen hier aufef"uhrten Aspekten
gerecht wird.
%%%%%%%%%%%%%%%%%%%%%%%%%%%%%%%%%%%%%%%%%%%%%%%%%%%%%%%%%%%%%%%%%%%%%%%%%%%%%%%
%% StuPro B, "Programmierumgebung Offener Antrieb" (POA)
%% Spezifikation
%% $Id: produktziele.tex,v 1.1 2003/07/01 10:08:29 vanto Exp $
%% Achtung: Diese Datei wird in die Spec inkludiert!
%%%%%%%%%%%%%%%%%%%%%%%%%%%%%%%%%%%%%%%%%%%%%%%%%%%%%%%%%%%%%%%%%%%%%%%%%%%%%%%

\section {Produktziele}
\subsection{Hintergrund}
Es werden viele Neuentwicklungen im Bereich der Werkzeugmaschinen gemacht.
Viele dieser Entwicklungen beinhalten neue Kinematiken, wie z.B. die
Parallelkinematiken und Sensoren (z.B. den Ferraris 
Relativbeschleunigungssensor).\par
Dadurch entstehen neue Anforderungen an die Antriebsregelung. Zus�tzliche
Sensor-Signale m�ssen in den Reglerstrukturen ber�cksichtigt werden -- oder
es werden sogar v�llig neue Reglerstrukturen ben�tigt.\par
Die momentan auf dem Markt erh�ltlichen Reglersysteme erlauben meist
weder die Ber�cksichtigung neuartiger Sensoren, noch bieten sie die
M�glichkeit, eigene anwenderspezifische Reglerstrukturen zu implementieren.\par
Daher wird am ISW eine Plattform f�r die Antriebsregelung entwickelt,
auf der es dem Anwender in jeder Hinsicht offen steht, eigene Funktionalit�ten
zu integrieren. Diese Plattform wird am ISW kurz als ``Offener Antrieb''
bezeichnet.  \par
Die hardwaretechnische Realisierung erfolgt in Form einer PC-Einsteckplatine.
Zentrales Element des Offenen Antriebes ist der Altera ``APEX'' Baustein. Es
handelt sich dabei um ein CPLD\footnote{Complex Programmable Logic Device},
das sich frei programmieren l�sst. Der Anwender hat die M�glichkeit, die
Funktion des Bausteins seinen Bed�rfnissen anzupassen.\par
Um die Offenheit f�r jeden Anwender nutzbar zu machen, wird f�r das CPLD eine
Architektur festgelegt, die es erm�glicht, einzelne Funktionalit�ten in Form
von Modulen zu implementieren. Diese Module k�nnen aus festprogrammierten
Schaltungen (Cores) und freiprogrammierbaren CPUs bestehen. Jedes Modul kann
auf die Signale aller anderen Module zugreifen und stellt seine eigenen
Ausgangssignale allen anderen Modulen zur Verf�gung.\par
Inhalt dieses Angebotes ist es, eine Programmierumgebung f�r das Netzwerk
von CPUs und Cores zu schaffen, in dem ein bereits auf dem CPLD vorhandenes
Netzwerk konfiguriert werden kann.

%%%%%%%%%%%%%%%%%%%%%%%%%%%%%%%%%%%%%%%%%%%%%%%%%%%%%%%%%%%%%%%%%%%%%%%%%%%%%%%
%% Funktionale Anforderungen
\subsection{Funktionen}

Die ``Programmierumgebung Offener Antrieb'' soll eine 
grafische Oberfl�che mit folgenden wesentlichen Funktionalit�ten bieten:
\begin{itemize}
\item Darstellung und Manipulation rasterisierter CPLD Layouts
\item Verwaltung und Editierung einer CPLD-Modulbibliothek zur CPLD-Layout
      Manipulation
\item Rahmencodegenerierung f�r eingebette CPU-Module in einem CPLD-Layout
\item Plausibilit�tspr�fung und Optimierung eines entworfenen CPLD-Layouts
\item Compilieren und Herunterladen von Quellcode f�r die CPU-Module
\item Speichern und �ffnen von CPLD-Layouts und zugeh�rigem Quellcode
\item Konfiguration der Programmeigenschaften
\item Zusammenarbeit mit externen Programmen
\end{itemize}
Auf die Einzelheiten der wesentlichen Funktionalit�ten und der Abgrenzung der
aufgeworfenen Begriffe wird in den folgenden Abschnitten eingegangen.

%%%%%%%%%%%%%%%%%%%%%%%%%%%%%%%%%%%%%%%%
%% CPLD Layouts
\subsubsection{CPLD Layouts}
Ein CPLD Layout ist ein virtuelles Abbild der realen Konfiguration
(Vernetzung der CPLD-Funktionsbl�cke) eines CPLD Hardware-Bausteins.
Dabei besteht die virtuelle Abbildung aus Funktionsbl�cken (CPUs, Cores,
E/A-Bl�cke) und Signalleitungen.\par
Im zu entwickelnden System soll ein CPLD Layout rasterisiert dargestellt
werden und durch Maus- und Tastatureingaben manipulierbar sein. Dabei
werden Funktionsbl�cke mit der Maus in ein Raster eingebettet und durch
einzelne Linien, die die Signalleitungen symbolisieren, verbunden.
Die Funktionsbl�cke selbst sollen konfigurierbar und editierbar sein.\par
CPLD-Layouts sollen XML-basiert speicher- und ladbar sein. Sie werden
zusammen mit dem zugeh�rigen Source- und evtl. vorhandenen Object-Code
in einer geeigneten Verzeichnisstruktur abgelegt.


%%%%%%%%%%%%%%%%%%%%%%%%%%%%%%%%%%%%%%%%
%% CPLD-Modulbibliothek
\subsubsection{CPLD-Modulbibliothek}
Die Modulbibliothek ist eine CPLD-Layout �bergreifende Sammlung von
Funktionsbl�cken (CPU, Cores, E/A-Bl�cke), die in CPLD-Layouts
einbettbar sein sollen. Die Funktionsbl�cke sollen in einer
XML-Struktur verwaltet werden. Die Modulbibliothek soll bei jedem
Systemaufruf automatisch geladen werden.

%%%%%%%%%%%%%%%%%%%%%%%%%%%%%%%%%%%%%%%%
%% Funktionsblock
\subsubsection{Funktionsblock}
Ein Funktionsblock ist ein virtuelles Abbild eines CPU-, Core- oder
E/A-Bausteins auf einem CPLD. Dabei sollen an allen Funktionsbl�cken
die Ein- und Ausg�nge mit Signalleitungen verbunden werden k�nnen.
\par
Speziell bei CPU-Bl�cken soll �berdies die Anzahl der Ein- und
Ausg�nge, die Benennung der E/As, das Einstellen eines Zeitoffsets,
das Setzen einer logischen Taktrate konfigurierbar sein und der auf
dem jeweiligen CPU-Block auszuf�hrende Sourcecode editierbar sein.

%%%%%%%%%%%%%%%%%%%%%%%%%%%%%%%%%%%%%%%%
%% Rahmencodegenerierung
\subsubsection{Rahmencodegenerierung}
Zur Editierung von C-Quellcode f�r CPU-Bl�cke soll ein vom Auftraggeber
vordefiniertes C-Template kontextabh�ngig f�r die existierende
Konfiguration eines CPU-Blocks verwendet werden, um Rahmencode zu generieren.
Dabei soll die Benennung der Signal-Ein- und Signal-Ausg�nge im C-Sourcecode
eines einzelnen CPU-Blocks angepa�t werden. Hierzu soll ein
Textersetzungsalgorithmus auf der Basis von Templates verwendet werden.



%%%%%%%%%%%%%%%%%%%%%%%%%%%%%%%%%%%%%%%%%%%%%%%%%%%%%%%%%%%%%%%%%%%%%%%%%%%%%%%
%% StuPro B, "Programmierumgebung Offener Antrieb" (POA)
%% Handbuch
%% $Id: definitionen.tex,v 1.3 2004/03/19 00:08:54 neco Exp $
%% Achtung: Diese Datei wird in das Handbuch inkludiert!
%%%%%%%%%%%%%%%%%%%%%%%%%%%%%%%%%%%%%%%%%%%%%%%%%%%%%%%%%%%%%%%%%%%%%%%%%%%%%%%
\newcommand{\begriff}[2]
{\item \bfseries{#1} \textnormal{#2}}

\chapter {Definitionen}
\begin{itemize}

\begriff{Ausgang}{Eine Speicheradresse auf die schreibend zugegriffen
  wird.}

\begriff{Block}{siehe Funktionsblock}

\begriff{Bus}{Mehrere Signalleitungen k"onnen zu einem Bus zusammen
  gefasst werden.}

\begriff{CPLD-Layout}{Anordnung von Funktionsbl"ocken und Signalleitungen.}

\begriff{CPLD}{Complex Programmable Logic Device}

\begriff{CPU}{Ein CPU-Block (auf Basis von NIOS).}

\begriff{Core}{Eine festgelegte Logik mit bekannter Laufzeit.}

\begriff{Download}{Herunterladen eines vollst"andig kompilierten CPLD-Layouts
auf das CPLD.}

\begriff{Eingang}{Eine Speicheradresse auf die lesend zugegriffen wird.}

\begriff{Episodisches Signal}{Ein Eingang oder Ausgang.}

\begriff{Funktionsblock}{Virtuelles Abbild eines Cores, einer CPU oder
eines Ein- bzw. Ausgabeblocks.}

\begriff{ISWOS}{Minimal-Betriebssystem, dass nach der Initialisierung
  des CPLDs auf den CPUs ausgef"uhrt wird.}

\begriff{Laufzeitermittlung}{Absch"atzung der maximalen Laufzeit eines
  Programms in Milli-Sekunden.}

\begriff{Offset}{Anzahl Takte bevor ein Funktionsblock zum ersten Mal
  angestossen wird.}

\begriff{Optimierung}{Verk"urzung der Gesamtlaufzeit eines CPLD-Zyklus.}

\begriff{Periode}{Anzahl Takte nach denen ein Funktionsblock erneut
  gestartet wird}

\begriff{Plausibilit"aspr"ufung}{Pr"ufung des CPLD-Layouts auf
  unverkn"upfte und falsch verkn"upfte Eingangs/Ausgangspins.}

\begriff{Scheduler}{Hardware, die Funktionsbl"ocke periodisch startet}

\begriff{Signalleitung}{Eine Verbindung zwischen zwei
  Funktionsbl"ocken. Eine Signalleitung hat immer 32-bit.}

\begriff{Takt}{Kleinste Zeiteinheit des Schedulers}

\begriff{Verbindung}{Eine Kopplung von einem Ausgang zu ein oder
  mehreren Eing"angen.}

\end{itemize}



%%%%%%%%%%%%%%%%%%%%%%%%%%%%%%%%%%%%%%%%%%%%%%%%%%%%%%%%%%%%%%%%%%%%%%%%%%%%%%%
%% Produkteinsatz
%\include {produkteinsatz}

%%%%%%%%%%%%%%%%%%%%%%%%%%%%%%%%%%%%%%%%%%%%%%%%%%%%%%%%%%%%%%%%%%%%%%%%%%%%%%%
%% Produktuebersicht
\chapter {Produkt"ubersicht}
Hier usecase diagramm?

%%%%%%%%%%%%%%%%%%%%%%%%%%%%%%%%%%%%%%%%%%%%%%%%%%%%%%%%%%%%%%%%%%%%%%%%%%%%%%%
%% Usecases
%%%%%%%%%%%%%%%%%%%%%%%%%%%%%%%%%%%%%%%%%%%%%%%%%%%%%%%%%%%%%%%%%%%%%%%%%%%%%%%
%% StuPro B, "Programmierumgebung Offener Antrieb" (POA)
%% Spezifikation
%% $Id: usecases.tex,v 1.1 2003/07/01 10:08:29 vanto Exp $
%% Achtung: Diese Datei wird in die Spec inkludiert!
%%%%%%%%%%%%%%%%%%%%%%%%%%%%%%%%%%%%%%%%%%%%%%%%%%%%%%%%%%%%%%%%%%%%%%%%%%%%%%%

\newcommand{\usecase}[5]{
\nopagebreak
\textbf{Vorbedingung}\\
#1 \\
\textbf{Beschreibung}\\
#2 \\
\textbf{Nachbedingung Erfolg}\\
#3 \\
\textbf{Nachbedingung Fehlschlag}\\
#4 \\
\textbf{Interaktionen}\\
#5\\
}

\chapter {Produktfunktionen}
\section {Akteure}
POA unterscheidet keine Akteure, somit gibt es nur den POA-Benutzer
\subsection {AKT Benutzer}
Der POA-Benutzer ist in der Regel ein C-Programmierer.
\section {Use-Cases}
Die folgenden Gesch�ftsprozesse beschreiben die Funktionalit�t des Prototyps

\subsection {UC Neues Projekt anlegen}
\usecase 	{Programm ist gestartet}
		{Der Benutzer w�hlt im Men� \emph{File} den Eintrag \emph{New Project...}. Es �ffnet sich ein
		Dialog, in dem der Benutzer den Namen des neuen Projektes angibt und das Wurzelverzeichnis ausw�hlt.}
		{Es wird unterhalb des Wurzelverzeichnisses ein neues Projekt angelegt. Dazu wird die unter 5.2 
		angegebene Verzeichnisstruktur erzeugt. Das neue Projekt wird ge�ffnet.}
		{Es ist kein Projekt ge�ffnet.}
		{AKT Benutzer}
					
\subsection {UC Projekt �ffnen}
\usecase 	{Programm ist gestartet}
		{Der Benutzer w�hlt im Men� \emph{File} den Eintrag \emph{Open Project...}. Es �ffnet sich ein 
		�ffnen-Dialog, in dem der Benutzer die project.xml-Datei in dem gew�nschten Projektverzeichnis 
		ausw�hlt und best�tigt.}
		{Das Projekt wird ge�ffnet. Es wird ein neues MDI-Hauptfenster angezeigt in dem sich Funktionsbl�cke
		und deren Verbindungen angezeigt werden.}
		{Es ist kein Projekt ge�ffnet.}
		{AKT Benutzer}
					
\subsection {UC Projekt speichern}
\usecase 	{Programm ist gestartet, es ist ein Projekt ge�ffnet}
		{Der Benutzer w�hlt im Men� \emph{File} den Eintrag \emph{Save} aus.}
		{�nderungen am Projekt werden gespeichert.}
		{Nicht definiert}
		{AKT Benutzer}
					
\subsection {UC Projekt kompilieren}
\usecase 	{Programm ist gestartet, es ist ein Projekt ge�ffnet}
		{Der Benutzer w�hlt im Men� \emph{File} den Eintrag \emph{Compile Project} aus.
		Das Projekt wird gespeichert. Danach wird die Sourcecodedatei von jeder CPU kompiliert.}
		{Das kompilierte Projekt-C-Code wird im Projektverzeichnis abgelegt.}
		{Nicht definiert}
		{AKT Benutzer}

\subsection {UC Projekt auf den CPLD downloaden}
\usecase 	{Programm ist gestartet, es ist ein Projekt ge�ffnet}
		{Der Benutzer w�hlt im Men� \emph{File} den Eintrag \emph{Download to CPLD-Board} aus.
		Das Projekt wird gespeichert und kompiliert.
		Das Downloadtool wird gestartet und beginnt den Download. Die Ausgabe des Compilers und des 
		Downloadtools wird einem Fenster angezeigt}
		{Das Projekt-C-Code wird auf den CPLD geschrieben.}
		{Nicht definiert.}
		{AKT Benutzer}

\subsection {UC Programmeinstellungen setzen}
\usecase 	{UC Programm ist gestartet}
		{Der Benutzer w�hlt im Men� \emph{File} den Eintrag \emph{Preferences}. Es �ffnet sich ein 
		Dialog in dem der Benutzer die Programmeinstellungen vornehmen kann. 
		Dazu geh�rt:
		\begin{itemize}
		  \item Pfad und Dateiname des externen Editors
		  \item Pfad und Dateiname des externen Compilers
		  \item Pfad des Verzeichnisses, das die C-Code-Vorlagen enth�lt
		  \item Pfad und Dateiname des Downloadtools
		\end{itemize}
		Der Dialog kann mit \emph{OK} oder \emph{Cancel} geschlossen werden.}
		{Ge�nderte Programmeinstellungen werden gepeichert}
		{Ge�nderte Programmeinstellungen werden nicht gepeichert}
		{AKT Benutzer}

\subsection {UC Funktionsblock in das Projekt einf�gen}
\usecase 	{UC Programm ist gestartet, es ist ein Projekt ge�ffnet}
		{Der Benutzer w�hlt einen Funktionsblock aus der Bibliothek und zieht ihn per Drag\&Drop auf einen 
		freien Platz in das Projektfenster. Ist er dort abgelegt, �ffnet sich der \emph{Preferences}-Dialog
		f�r den neuen Block.}
		{Funktionsblock wird ein das Projekt eingef�gt. Ist der Funktionsblock eine CPU, so wird 
		automatisch der Rahmencode generiert.}
		{Nicht definiert.}
		{AKT Benutzer}
					
\subsection {UC Funktionsblock aus dem Projekt entfernen}
\usecase 	{UC Programm ist gestartet, es ist ein Projekt ge�ffnet}
		{Der Benutzer w�hlt markiert mit der Maus einen Funktionsblock und dr�ckt die \emph{Entf}-Taste oder
		w�hlt \emph{Remove} im Kontextmen� aus. Ist der Funktionsblock bereits mit anderen Funktionsbl�cken
		verbunden, muss der Benutzer den L�schvorgang explizit best�tigen. Dann werden zus�tzlich alle 
		Verbindungen mit diesem Block gel�scht.}
		{Funktionsblock wird entfernt.}
		{Funktionsblock bleibt im Projekt enthalten.}
		{AKT Benutzer}

\subsection {UC Eigenschaften eines Funktionsblock setzen}
\usecase 	{UC Programm ist gestartet, es ist ein Projekt ge�ffnet}
		{Der Akteur �ffnet den \emph{Preferences}-Dialog �ber das Context-Men� des 
		Funktionsblocks oder mittels Doppelklick. Hier kann er die folgenden Einstellungen machen und 
		beendet den Dialog mittels \emph{OK} oder \emph{Cancel} (Fehlschlag). Beim Schlie�en �ber \emph{OK} 
		werden die Eingabefelder validiert, existieren noch tempor�re Eintr�ge im IO-Baum, so m�ssen diese
		zuerst entfernt oder berichtigt werden.}
		{Die Einstellungen werden �bernommen. Ist dieser Funktionsblock eine CPU, so werden die Konstanten,
		die die I/O-Adressen enthalten automatisch angepasst bzw. erg�nzt.}
		{Die Einstellungen werden verworfen}
		{AKT Benutzer}

\subsubsection {UC Name des Funktionsblocks setzen}
\usecase 	{UC Programm ist gestartet, es ist ein Projekt ge�ffnet und der \emph{Preferences}-Dialog 
		eines Funktionsblocks ist sichtbar}
		{In dem Feld \emph{Name} weist der Benutzer dem Funktionsblock einen eindeutigen Bezeichner zu. 
		Im Projektkontext gilt dieser Bezeichner als Prefix f�r die I/O-Variablen des Funktionsblocks.}
		{Nicht definiert}
		{Nicht definiert}
		{AKT Benutzer}

\subsubsection {UC Adresse der CPU setzen}
\usecase 	{UC Programm ist gestartet, es ist ein Projekt ge�ffnet und der \emph{Preferences}-Dialog 
		einer CPU ist sichtbar}
		{In dem Feld \emph{Address} gibt der Benutzer die Basisadresse der CPU an.}
		{Nicht definiert}
		{Nicht definiert}
		{AKT Benutzer}

\subsubsection {UC Takt-Offset der CPU setzen}
\usecase 	{UC Programm ist gestartet, es ist ein Projekt ge�ffnet und der \emph{Preferences}-Dialog 
		einer CPU ist sichtbar}
		{In dem Feld \emph{Offset} gibt der Benutzer den Takt-Offset (in Takten des Systemtakts?) an, 
		um die die Ausf�hrung des Programmcodes verschoben wird.}
		{Nicht definiert}
		{Nicht definiert}
		{AKT Benutzer}

\subsubsection {UC Takt der CPU setzen}
\usecase 	{UC Programm ist gestartet, es ist ein Projekt ge�ffnet und der \emph{Preferences}-Dialog 
		einer CPU ist sichtbar}
		{In dem Feld \emph{Clock} gibt der Benutzer den Takt (in Takten des Systemtakts?, in MS?) an, 
		in welchem die CPU ihren Programmcode ausf�hren soll.}
		{Nicht definiert}
		{Nicht definiert}
		{AKT Benutzer}
		
\subsubsection {UC Laufzeit der CPU setzen}
\usecase 	{UC Programm ist gestartet, es ist ein Projekt ge�ffnet und der \emph{Preferences}-Dialog 
		einer CPU ist sichtbar}
		{In dem Feld \emph{Runtime} gibt der Benutzer die Laufzeit des Programmcodes (in Takten des CPU-Takt) an. 
		Es wird automatisch ein Wert vorgeschlagen, den das Programm �ber eine Schnittstelle bezieht.}
		{Nicht definiert}
		{Nicht definiert}
		{AKT Benutzer}

\subsubsection {UC Ein-/Ausgang hinzuf�gen}
\usecase 	{UC Programm ist gestartet, es ist ein Projekt ge�ffnet und der \emph{Preferences}-Dialog 
		einer CPU ist sichtbar}
		{Der Benutzer w�hlt eines der Wurzelelemente im IO-Baum aus (\emph{Input},\emph{Output}) und 
		klick auf den \emph{Add}-Button. Es wird ein Eintrag unter dieser Kategorie hinzugef�gt. Dieser
		ist vorerst tempor�r, solange er noch keinen CPU-internen uniquen Bezeichner und eine Speicher-Addresse??
		erhalten hat. Im Projektkontext wird 
		dem Bezeichner der Name des Funktionsblocks als Prefix gesetzt und mit einem "'\_"' verbunden.}
		{Nicht definiert}
		{Nicht definiert}
		{AKT Benutzer}

\subsubsection {UC Ein-/Ausgang entfernen}
\usecase 	{UC Programm ist gestartet, es ist ein Projekt ge�ffnet und der \emph{Preferences}-Dialog 
		einer CPU ist sichtbar}
		{Der Benutzer w�hlt ein I/O-Element im IO-Baum aus und 
		klick auf den \emph{Remove}-Button. Das Element wird entfernt, sofern keine andere CPU von diesem
		Ein-/Ausgang gebrauch macht.}
		{Nicht definiert}
		{Nicht definiert}
		{AKT Benutzer}

\subsubsection {UC I/O-Adresse eines Ein-/Ausgangs �ndern??}
\usecase 	{UC Programm ist gestartet, es ist ein Projekt ge�ffnet und der \emph{Preferences}-Dialog 
		einer CPU ist sichtbar}
		{Der Benutzer w�hlt ein I/O-Element im IO-Baum aus und startet mittel Doppelklick in der 
		\emph{Address}-Spalte den Editor. Dort gibt er nun die I/O-Adresse...}
		{Nicht definiert}
		{Nicht definiert}
		{AKT Benutzer}

\subsection {UC Funktionsblock in dem Projekt verbinden}
\usecase 	{UC Programm ist gestartet, es ist ein Projekt ge�ffnet}
		{Der Benutzer �ffnet das Kontextmenu von einem Ein-/Ausgang eines Funktionsblocks und w�hlt den Eintrag
		\emph {Connect}. Von nun an folgt eine rechtwinklige Line dem Mauszeiger bis der Benutzer den 
		Ein-/Ausgang eines anderen Funktionsblocks anklickt. Der Vorgang kann durch dr�cken der \emph{ESC}-Taste
		abgebrochen werden. Klickt der Benutzer mit der linken Maustaste auf eine freie Stelle im Projektfenster,
		so wird ein Zwischenpunkt eingef�gt, von dem aus rechtwinklig die Linie fortgesetzt wird. Mit einen 
		Rechtsklick wird der jeweils letzte Zwischenpunkt entfernt. Das Einf�gen und Entfernen von Zwischenknoten
		ist nach dem Erreichen/Verbinden des Zielfunktionsblocks nicht mehr m�glich.}
		{Die Verbindung wird hergestellt.}
		{Es wird keine Verbindung hergestellt.}
		{AKT Benutzer}

\subsection {UC Sourcecode einer CPU bearbeiten}
\usecase 	{UC Programm ist gestartet, es ist ein Projekt ge�ffnet, eine CPU ist selektiert}
		{Der Benutzer w�hlt im Kontextmen� der markierten CPU den Eintrag \emph{Edit Source...}. Darauf 
		hin wird der externe Editor gestartet.}
		{Nicht definiert}
		{Nicht definiert}
		{AKT Benutzer}



%%%%%%%%%%%%%%%%%%%%%%%%%%%%%%%%%%%%%%%%%%%%%%%%%%%%%%%%%%%%%%%%%%%%%%%%%%%%%%%
%% Usecases
%%%%%%%%%%%%%%%%%%%%%%%%%%%%%%%%%%%%%%%%%%%%%%%%%%%%%%%%%%%%%%%%%%%%%%%%%%%%%%%
%% StuPro B, "Programmierumgebung Offener Antrieb" (POA)
%% Spezifikation
%% $Id: produktdaten.tex,v 1.2 2003/07/04 14:40:01 garbeam Exp $
%% Achtung: Diese Datei wird in die Spec inkludiert!
%%%%%%%%%%%%%%%%%%%%%%%%%%%%%%%%%%%%%%%%%%%%%%%%%%%%%%%%%%%%%%%%%%%%%%%%%%%%%%%

\chapter {Produktdaten}
\section {Projektdatei}
%\lstset{language=XML}
\begin{verbatim}
<project>
  <meta>
    <name/>
    <description/>
    <author/>
    <created/>
    <changed/>
  </meta>

  <cores>
    <core name="SuperCore" icon="icon" runtime="12ms">
      <input id="pin01" name="Pin 1" address="0x810" width="1"/> 
      <!-- ??? sind die addressen statisch in der bib, oder dynamisch abhaengig von der instanz? -->
      <input id="pin02to34" name="Pin 2 bis 34" address="0x812" width="32"/>
      <output id="pin03" name="Pin 3" address="0x814" width="2"/>
    </core>
  </cores>
  <cpus>
    <cpu name="SuperCPU" icon = "icon">
  </cpus>

  <instances>
    <core ref = "SuperCore" id = "myscore" address="0x100"/>
    <core ref = "SuperCore" id = "mysecondscore" address="0x200"/>
    <cpu ref = "SuperCPU" id = "bootCpu" runtime="120clks clock="6Mhz" address="0x300">
      <input id="pin01" name="Pin 1" address="0x310" width="1"/>
    </cpu>
  </instances>
</project>
\end{verbatim}

\section{Verzeichnissstuktur}

F�r jedes Projekt wird ein Verzeichniss mit einer festgelegten
Struktur erzeugt.

\begin{itemize}
\item CPU\_NR\_sdk
  \begin{itemize}
  \item lib\\
    ?
  \item src \\
    Source Code
  \item inc \\
    NIOS Header Dateien
  \end{itemize}
\item poa
  \begin{itemize}
  \item project.xml
  \end{itemize}
\end{itemize}

Innerhalb des Projektverzeichnisses wird zum Kompilieren ``nb'' 
gestartet. 



%%%%%%%%%%%%%%%%%%%%%%%%%%%%%%%%%%%%%%%%%%%%%%%%%%%%%%%%%%%%%%%%%%%%%%%%%%%%%%%
%% QS
\chapter {Qualit"atsanforderungen}

%%%%%%%%%%%%%%%%%%%%%%%%%%%%%%%%%%%%%%%%%%%%%%%%%%%%%%%%%%%%%%%%%%%%%%%%%%%%%%%
%% GUI
\chapter {Benutzungsoberfläche}

%%%%%%%%%%%%%%%%%%%%%%%%%%%%%%%%%%%%%%%%%%%%%%%%%%%%%%%%%%%%%%%%%%%%%%%%%%%%%%%
%% Non functional requirements
\chapter{Nichtfunktionale Anforderungen}
\section{Leistungsanforderungen}
TODO: Mengenger"ust, Kommunikation, Synchronisation der Daten
\section{Entwurfseinschr"ankungen}
Die Programmiersprache ist C++, als Klassenbibliothekt soll QT\footnote{Trolltech} genutzt werden. Desweiteren k"onnen weitere 
Bibliotheken unter Open Source Lizenz verwendet werden.

\section{Verf"ugbarkeit}
Keine besonderen Anforderungen bzgl. Verf"ugbarkeit.

\section{Sicherheit}
Keine besonderen Anforderungen bzgl. Datensicherheit.
\section{Robustheit}
TODO: Robustheit gegen"uber fehlerhaften Eintr"agen, Validierung etc
\section{Wartbarkeit}
TODO: Sauberer Entwurf ?
\section{Portabilit"at}
POA soll sowohl unter Windows als auch unter Linux lauffähig sein. Diese Anforderung wird von der QT-Bibliotek gewährleistet.
Zur Abnahme des Produktes muss POA jedoch nur unter Linux kompilier- und ausfürbar sein.


%%%%%%%%%%%%%%%%%%%%%%%%%%%%%%%%%%%%%%%%%%%%%%%%%%%%%%%%%%%%%%%%%%%%%%%%%%%%%%%
%% Entwicklungsumgebung
\chapter {Anforderungen an die Entwicklungsumgebung}
Die Implementierung erfolgt in GNU C++ auf der Basis von Trolltech Qt 3.1.1. Zielplattform wird Redhat GNU/Linux 8.2 sein.

\end{document}
