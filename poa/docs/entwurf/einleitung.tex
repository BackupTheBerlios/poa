%%%%%%%%%%%%%%%%%%%%%%%%%%%%%%%%%%%%%%%%%%%%%%%%%%%%%%%%%%%%%%%%%%%%%%%%%%%%%%%
%% StuPro B, "Programmierumgebung Offener Antrieb" (POA)
%% Entwurf
%% $Id: einleitung.tex,v 1.1 2003/07/13 20:15:58 garbeam Exp $
%% Achtung: Diese Datei wird in den Entwurf inkludiert!
%%%%%%%%%%%%%%%%%%%%%%%%%%%%%%%%%%%%%%%%%%%%%%%%%%%%%%%%%%%%%%%%%%%%%%%%%%%%%%%

\chapter {Einleitung}

Dieses Dokument ist Teil des Entwurfes von POA und kennzeichnet die
Design-Entscheidungen f�r den Prototyp ''Messe''. Als Grundlage f�r den Umfang
des Prototyps dient die interne Spezifikation.
\par
Hervorgerufen durch das agile Prozessmodell der POA-Entwicklung, ergeben sich
an den Entwurf von POA folgende verschiedene Anforderungen:
\begin{itemize}
\item leichte Erweiterbarkeit
\item einfaches Design
\item schnelle Integrierbarkeit
\item hohe Modularit�t
\end{itemize}
\par
Um diesen Anforderungen gerecht zu werden, wurden geeignete
Design-Entscheidungen getroffen, die in Kapitel 2 in allgemeiner Form
erl�utert werden.
\par
Weiterhin wurde f�r den Entwurf Rahmen-Sourcecode in Form von Header Dateien
erzeugt, der als Ausgangspunkt zur Implementierung des Prototypen ''Messe''
genutzt werden soll, da er bereits alle wesentlichen Design-Entscheidungen
enth�lt.\par
In Kapitel 3 wird der konkrete Entwurf kurz erl�utert.

%%% Local Variables: 
%%% TeX-master: "angebot"
%%% End: 
%%% vim:tw=79:
