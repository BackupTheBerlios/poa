%%%%%%%%%%%%%%%%%%%%%%%%%%%%%%%%%%%%%%%%%%%%%%%%%%%%%%%%%%%%%%%%%%%%%%%%%%%%%%%
%% StuPro B, "Programmierumgebung Offener Antrieb" (POA)
%% Entwurf
%% $Id: design.tex,v 1.3 2004/06/04 14:43:18 squig Exp $
%% Achtung: Diese Datei wird in den Entwurf inkludiert!
%%%%%%%%%%%%%%%%%%%%%%%%%%%%%%%%%%%%%%%%%%%%%%%%%%%%%%%%%%%%%%%%%%%%%%%%%%%%%%%

\section{Entwurfsmuster}

Während des Entwurfs von POA konnte ein Teil der Implementierung mit
Hilfe von Entwufsmustern der Softwaretechnik realisiert
werden. Entwufsmuster beschreiben eine Lösungsstrategie für ein
abstraktes Problem.

\subsection{Model-View-Controller}

Die Speicherung der Daten ist vollständig von der Visualisierung
unabhängig. Die Daten werden in sogenannten Model Klassen verwaltet
und von View Klassen angezeigt, wobei eine 1:n Beziehung besteht. Die
Kommunikation zwischen den Klassen erfolgt anonym über
Benachrichtigungen (siehe auch \ref{Benachrichtigungen}). 

Eine Model Klasse stellt über get-Methoden oder Iteratoren den Zustand
zur Verfügung. Dieser Zustand wird von einer View-Klasse dem
Anwendungszweck entsprechend angezeigt. Konkret werden die Blöcke,
Pins und Verbindungen in POA jeweils durch Datenobjekte
repräsentiert. Diese werden mit Hilfe von entsprechenden Viewobjekten von der
GridCanvas Klasse angezeigt. 

\subsection{Benachrichtigung}

Die QT Bibliothek erweitert den Sprachumfang von C++ über einen
speziellen Kompiler, den Meta-Object-Compiler (MOC), mit einem
Benachrichtigungskonzept. Mit Hilfe von
Benachrichtigungen können Objekte anderen Objekten Mitteilungen über
Zustandsänderungen schicken. Die nahtlose Integration in die
Programmiersprache ermöglicht eine besonders einfache Verwendung des
Mechanismus.

Die Benachrichtigungen (Signals) werden durch das Schlüsselwort emit
gesendet und von speziellen Methoden (Slots) empfangen. Die Signals
und Slots werden in den Klassendeklarationen angegeben. Die Verbindung
zwischen zwei Objekten wird mittels der connect Methode hergestellt. 

In POA wird an meheren Stellen gebrauch von dem Signal-Slot
Mechanismus gemacht.

\subsection{Singleton}

Das Singleton Muster wird verwendet, wenn ein Objekt nur genau einmal
instanziert werden soll. Dadurch lassen sich auch in
objekt-orientierten Systemen, äquivalent zu globalen Variablen in
imperativen Systemen, global verfügbare Objekte realisieren,
die über eine definierte Schnittstelle angesprochen werden können. 

In POA wird nur an e


\section{Module}

Die Architektur von POA lässt sich gut in stark zusammenhängende
Module aufteilen, die eine gerine Kopplung aufweisen. In den folgenden
Abschnitten sind diese Module im Detail beschrieben.

Die Zuordnung von einzelnen Klassen zu Modulen lässt sich in der Regel
aus dem Klassennamen ableiten oder ist aus dem POA
Software-Architektur anhand der Schattierung ersichtlich.

\subsection{Datenmodell (Model)}

Die Klassen in dem 

\subsection{View}

\subsection{Scheduling}

Das Scheduling Modul 

\subsection{Download}

\subsection{Routing}

\subsection{Problemreport}

\subsection{Querschnittsmodule}



\section{Persistenz}

\subsection{XML Format}




%%% Local Variables: 
%%% TeX-master: "angebot"
%%% End: 
%%% vim:tw=79:
