%%%%%%%%%%%%%%%%%%%%%%%%%%%%%%%%%%%%%%%%%%%%%%%%%%%%%%%%%%%%%%%%%%%%%%%%%%%%%%%
%% StuPro B, "Programmierumgebung Offener Antrieb" (POA)
%% Angebot
%% $Id: begriffslexikon.tex,v 1.7 2003/06/17 23:06:24 squig Exp $
%%%%%%%%%%%%%%%%%%%%%%%%%%%%%%%%%%%%%%%%%%%%%%%%%%%%%%%%%%%%%%%%%%%%%%%%%%%%%%%
\newcommand{\begriff}[2]
{\item \bfseries{#1} \textnormal{#2}}

\chapter{Begriffslexikon}
\begin{itemize}

\begriff{Block}{siehe Funktionsblock}

\begriff{Bus}{Mehrere Signalleitungen k�nnen zu einem Bus zusammen
  gefasst werden.}

\begriff{Core}{Eine festgelegte Logik mit bekannter Laufzeit.}

\begriff{CPLD}{Complex Programmable Logic Device}

\begriff{CPLD-Layout}{Anordnung von Funktionsbl�cken und Signalleitungen.}

\begriff{CPU}{Ein CPU-Block (auf Basis von NIOS).}

\begriff{Download}{Herunterladen eines vollst�ndig compilierten CPLD-Layouts
auf das CPLD.}

\begriff{Funktionsblock}{Virtuelles Abbild eines Cores, einer CPU oder
eines Ein- bzw. Ausgabeblocks.}

\begriff{Laufzeitermittlung}{Absch�tzung der maximalen Laufzeit eines
  Programms in Milli-Sekunden.}

\begriff{ISWOS}{Minimal-Betriebssystem, dass nach der Initialisierung
  des CPLDs auf den CPUs ausgef�hrt wird.}

\begriff{Plausibilit�tspr�fung}{Pr�fung des CPLD-Layouts auf
  unverk�pfte und falsch verkn�pfte E/As.}

\begriff{Optimierung}{Verk�rzung der Gesamtlaufzeit eines CPLD-Zyklus.}

\begriff{Signalleitung}{Eine Verbindung zwischen zwei
  Funktionsbl�cken. Eine Signalleitung hat immer 32-bit.}

\begriff{Verbindung}{Eine Kopplung von einem Ausgang zu ein oder
  mehreren Eing�ngen.}

\end{itemize}

%%% Local Variables: 
%%% TeX-master: "angebot"
%%% End: 
%%% vim:tw=79:
