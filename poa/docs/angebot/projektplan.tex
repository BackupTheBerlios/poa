%%%%%%%%%%%%%%%%%%%%%%%%%%%%%%%%%%%%%%%%%%%%%%%%%%%%%%%%%%%%%%%%%%%%%%%%%%%%%%%
%% StuPro B, "Programmierumgebung Offener Antrieb" (POA)
%% Projektplan
%% $Id: projektplan.tex,v 1.5 2003/06/12 11:35:47 papier Exp $
%% Achtung: Diese Datei wird ins Angebot inkludiert!
%%%%%%%%%%%%%%%%%%%%%%%%%%%%%%%%%%%%%%%%%%%%%%%%%%%%%%%%%%%%%%%%%%%%%%%%%%%%%%%

\newcommand{\milestone}[4]{
\paragraph* {#1} { (#2) }
\\[0.3cm]
Ergebnis: \begin {bfseries} #3 \end {bfseries}
\\[0.3cm]
#4
}

\chapter {Projektplan}

\section {Zeitplan}
Zu Anfang jedes Sprints werden in einer priorisierten Liste die
Ziele f�r den Sprint festgelegt und innerhalb von 2 Wochen fertig
gestellt. 

\section {Kostensch�tzung}

\section {Meilensteine}
\label {Meilensteine}

Das Projekt ist, wie aus dem Zeitplan ersichtlich, in mehrere Phasen
eingeteilt. Die Meilensteine dienen dazu, den erfolgreichen Abschluss
einer Phase zu �berpr�fen. Zum erfolgreichen Erreichen eines
Meilensteines m�ssen bestimmte Dokumente oder Softwareprogramme
fertiggestellt werden. Diese sind bei den Meilensteinen unter Ergebnis
angegeben. Die Meilensteinabnahme ist erforderlich, um zur n�chsten 
Phase �berzugehen bzw. eine auf den Ergebnissen aufbauende T�tigkeit 
durchzuf�hren. Dies wird nicht getan, bis die Meilensteinabnahme erfolgt ist.

Die erstellten Dokumente oder Softwareprogramme werden dem
Auftraggeber vorgelegt. Sollten die erzielten Ergebnisse nicht
den Erwartungen entsprechen, m�ssen diese nachgebessert
werden. Sollten dadurch erhebliche Verz�gerungen im Projekt auftreten,
muss der Zeitplan entsprechend angepasst werden. Der Auftraggeber nimmt
den Meilenstein ab.

Abbruchmeilensteine.

\milestone {M1: Angebot}{extern}{Angebot}
{Das vorliegende Dokument.}

\milestone {M2: Prototyp Pr�sentation}{extern}{Lauff�higer Prototyp}
{Der Prototyp, der auf der Messe pr�sentiert werden soll.}

\milestone {M3: Prototyp Endg�ltig}{extern}{Lauff�higer Prototyp}
{Ein mit Funktionalit�t erweiterter Prototyp.}

\milestone {M4: Auslieferung}{extern}{Software}
{Auslieferung des erstellten Systems.}

\section{Risiken}

%%% Local Variables: 
%%% TeX-master: "angebot"
%%% End: 
%%% vim:tw=79:
