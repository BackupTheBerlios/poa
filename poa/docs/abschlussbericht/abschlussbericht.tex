%%%%%%%%%%%%%%%%%%%%%%%%%%%%%%%%%%%%%%%%%%%%%%%%%%%%%%%%%%%%%%%%%%%%%%%%%%%%%%%
%% StuPro B, "Programmierumgebung Offener Antrieb" (POA)
%% Abschlussbericht
%% $Id: abschlussbericht.tex,v 1.2 2004/04/26 20:23:18 squig Exp $
%%%%%%%%%%%%%%%%%%%%%%%%%%%%%%%%%%%%%%%%%%%%%%%%%%%%%%%%%%%%%%%%%%%%%%%%%%%%%%%
\documentclass[a4paper,titlepage,12pt,ngerman]{scrbook}
\usepackage{../common/header}
\RCSdef $Revision: 1.2 $
\RCSdef $Date: 2004/04/26 20:23:18 $

\newcommand\version{Version 1.0 \xspace}

\begin{document}

%%%%%%%%%%%%%%%%%%%%%%%%%%%%%%%%%%%%%%%%%%%%%%%%%%%%%%%%%%%%%%%%%%%%%%%%%%%%%%%
%% Deckblatt

\begin{titlepage}
\renewcommand{\thefootnote}{\fnsymbol{footnote}}
{
\Huge
\raggedright
\textbf{POA} \\
\huge Programmierumgebung Offener Antrieb
\rule{\textwidth}{0.75pt}
\par
}
\begin{flushleft}
\normalsize
\version
\vfill

\begin{figure}[htbp]
\begin{center}
\includegraphics[width=15cm]{../common/poa-logo}
\end{center}
\end{figure}

\end{flushleft}
\vfill

\Huge Architektur

\setcounter{footnote}{0}
\end{titlepage}

%%%%%%%%%%%%%%%%%%%%%%%%%%%%%%%%%%%%%%%%%%%%%%%%%%%%%%%%%%%%%%%%%%%%%%%%%%%%%%%
%% Versionsgeschichte

\section*{Versionsgeschichte}

\begin{itemize}

\item Version 1.0 (xx.05.2004)

  Diese Version wurde dem Auftraggeber vorgelegt.

\end{itemize}

%%%%%%%%%%%%%%%%%%%%%%%%%%%%%%%%%%%%%%%%%%%%%%%%%%%%%%%%%%%%%%%%%%%%%%%%%%%%%%%
%% Inhaltsverzeichnis

\tableofcontents

%%%%%%%%%%%%%%%%%%%%%%%%%%%%%%%%%%%%%%%%%%%%%%%%%%%%%%%%%%%%%%%%%%%%%%%%%%%%%%%
\chapter{Einleitung}

\section{Zweck des Dokuments}

\section{Projektziele}

Es werden viele Neuentwicklungen im Bereich der Werkzeugmaschinen
gemacht.  Viele dieser Entwicklungen beinhalten neue Kinematiken, wie
z.B. die Parallelkinematiken und Sensoren (z.B. den Ferraris
Relativbeschleunigungssensor).  Dadurch entstehen neue Anforderungen
an die Antriebsregelung. Zus"atzliche Sensor-Signale m"ussen in den
Reglerstrukturen ber"ucksichtigt werden -- oder es werden sogar
v"ollig neue Reglerstrukturen ben"otigt.

Die momentan auf dem Markt erh"altlichen Reglersysteme erlauben meist
weder die Ber"ucksichtigung neuartiger Sensoren, noch bieten sie die
M"oglichkeit, eigene anwenderspezifische Reglerstrukturen zu
implementieren.

Daher wird am ISW eine Plattform f"ur die Antriebsregelung entwickelt,
auf der es dem Anwender in jeder Hinsicht offen steht, eigene
Funktionalit"aten zu integrieren. Diese Plattform wird am
ISW\footnote{Institut f"ur Steuerungstechnik der Werkzeugmaschinen und
  Fertigungseinrichtungen} an der Universit"at Stuttgart kurz als
``Offener Antrieb'' bezeichnet.

Die hardwaretechnische Realisierung erfolgt in Form einer Einsteckplatine.
Zentrales Element des Offenen Antriebes ist der Altera ``APEX'' Baustein. Es
handelt sich dabei um ein CPLD\footnote{Complex Programmable Logic Device},
das sich frei programmieren l"asst. Der Anwender hat die M"oglichkeit, die
Funktion des Bausteins seinen Bed"urfnissen anzupassen.

Um die Offenheit f"ur jeden Anwender nutzbar zu machen, wird f"ur das CPLD eine
Architektur festgelegt, die es erm"oglicht, einzelne Funktionalit"aten in Form
von Bl"ocken zu implementieren. Diese Bl"ocke k"onnen aus fest programmierten
Schaltungen (Cores) und freiprogrammierbaren CPUs bestehen. Jedes Block kann
auf die Signale aller anderen Bl"ocke zugreifen und stellt seine eigenen
Ausgangssignale allen anderen Bl"ocken zur Verf"ugung.

POA bietet eine anwenderfreundliche Programmierumgebung f"ur das Netzwerk
von CPUs und Cores, in dem ein bereits auf dem CPLD vorhandenes
Netzwerk konfiguriert werden kann.

\section{Anforderungen}

%%%%%%%%%%%%%%%%%%%%%%%%%%%%%%%%%%%%%%%%%%%%%%%%%%%%%%%%%%%%%%%%%%%%%%%%%%%%%%%
\chapter{Projektergebnisse}

\section{Programmierumgebung}
\subsection{Merkmale}

\section{Dokumente}

\subsection{Angebot}
\subsection{Handbuch}
\subsection{Interne Dokumente}

Entwurf, Spezifikation

%%%%%%%%%%%%%%%%%%%%%%%%%%%%%%%%%%%%%%%%%%%%%%%%%%%%%%%%%%%%%%%%%%%%%%%%%%%%%%%
\chapter{Projektorganisation}

\section{Projektrahmen}

Auftraggeber, Auftragnehmer

\section{Das Team}

\subsection{Rollen}

%%%%%%%%%%%%%%%%%%%%%%%%%%%%%%%%%%%%%%%%%%%%%%%%%%%%%%%%%%%%%%%%%%%%%%%%%%%%%%%
\chapter{Projektablauf}

\section{Prozessmodell}
\section{Projektplan}
\subsection{Anforderungsanalyse}
\subsection{Spezifikation}
\subsection{Entwurf}
\subsection{Prototyp}
\subsection{Erste Iterationen}
\subsection{Zweite Iterationen}
\subsection{GCover Unterprojekt}

\section{Meilensteine}

\section{Aufwand}

%%%%%%%%%%%%%%%%%%%%%%%%%%%%%%%%%%%%%%%%%%%%%%%%%%%%%%%%%%%%%%%%%%%%%%%%%%%%%%%
\chapter{Statistische Analyse}

\section{Versions Kontrolle}

\subsection{Evolution}

\section{Metriken}

\subsection{Test�berdeckung}

%%%%%%%%%%%%%%%%%%%%%%%%%%%%%%%%%%%%%%%%%%%%%%%%%%%%%%%%%%%%%%%%%%%%%%%%%%%%%%%
\chapter{Pers�nliche Berichte}

\section{Necati Aydin}
\section{Anselm Garbe}
\section{Stefan Hauser}
\section{Steffen Keul}
\section{Marcel Kilgus}
\section{Steffen Pingel}
\section{Tammo van Lessen}

\appendix

\chapter{Erkl�rung gem�� Pr�fungsordnung}

Ich erkl�re hiermit, da� ich im Rahmen meiner Mitarbeit am
Studienprojekt B - Programmierumgebung Offener Antrieb au�er von den
Betreuern vorgesehenen bzw. genehmigten Hilfsmitteln keine unzul��ige
Hilfe in Anspruch genommen habe.
\vspace{5cm}
Stuttgart, den xx. Mai 2004
\vspace{5cm}

\end{document}
