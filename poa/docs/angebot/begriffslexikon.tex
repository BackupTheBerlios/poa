%%%%%%%%%%%%%%%%%%%%%%%%%%%%%%%%%%%%%%%%%%%%%%%%%%%%%%%%%%%%%%%%%%%%%%%%%%%%%%%
%% StuPro B, "Programmierumgebung Offener Antrieb" (POA)
%% Angebot
%% $Id: begriffslexikon.tex,v 1.3 2003/06/12 11:35:47 papier Exp $
%%%%%%%%%%%%%%%%%%%%%%%%%%%%%%%%%%%%%%%%%%%%%%%%%%%%%%%%%%%%%%%%%%%%%%%%%%%%%%%
\newcommand{\begriff}[2]
{\item \bfseries{#1} \textnormal{#2}}

\chapter{Begriffslexikon}
\begin{itemize}

\begriff{Block}{Synonym: Funktionsblock}

\begriff{Bus}{Mehrere Signalleitungen, die von einem Funktionsblock A
  zu einem Funktionsblock B f�hren, werden der �bersichtlichkeit in
  der Darstellung zusammengefasst und als Bus bezeichnet.}

\begriff{Core}{Eine festgelegte Logik mit bekannter Laufzeit.}

\begriff{CPLD}{Complex Programmable Logic Device;
  Freibprogrammierbarer Baustein, auf den der Anwender mit Hilfe der
  zu erstellenden ``Prgrammierumgebung Offener Antrieb'' seinen
  Bed�rfnissen anpassen kann.}

\begriff{CPU}{Eine CPU auf Basis von NIOS.}

\begriff{Download}{Kopieren des compilierten CPLD-Layouts auf den CPLD.}

\begriff{Funktionsblock}{Ein Funktionsblock ist das virtuelle Abbild
  eines Cores, einer CPU oder eines Ein- bzw. Ausgabeblocks sein.}

\begriff{Plausibilit�tspr�fung}{Pr�fung des CPLD-Layouts auf
  unverk�pfte und falsch verkn�pfte E/As.}

\begriff{Optimierung}{Voraussetzung: plausibles CPLD-Layout; Es werden
  die Laufzeiten der CPU- und core-Bl�cke ermittelt und durch setzten
  von Offsets die Gesamtlaufzeit m�glichst verk�rzt.}


\end{itemize}

%%% Local Variables: 
%%% TeX-master: "angebot"
%%% End: 
%%% vim:tw=79:
