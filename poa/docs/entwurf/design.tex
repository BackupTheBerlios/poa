%%%%%%%%%%%%%%%%%%%%%%%%%%%%%%%%%%%%%%%%%%%%%%%%%%%%%%%%%%%%%%%%%%%%%%%%%%%%%%%
%% StuPro B, "Programmierumgebung Offener Antrieb" (POA)
%% Entwurf
%% $Id: design.tex,v 1.4 2004/06/04 19:28:54 squig Exp $
%% Achtung: Diese Datei wird in den Entwurf inkludiert!
%%%%%%%%%%%%%%%%%%%%%%%%%%%%%%%%%%%%%%%%%%%%%%%%%%%%%%%%%%%%%%%%%%%%%%%%%%%%%%%

\section{Entwurfsmuster}

W�hrend des Entwurfs von POA konnte ein Teil der Implementierung mit
Hilfe von Entwufsmustern der Softwaretechnik realisiert
werden. Entwufsmuster beschreiben eine L�sungsstrategie f�r ein
abstraktes Problem.

\subsection{Model-View-Controller}

Die Speicherung der Daten ist vollst�ndig von der Visualisierung
unabh�ngig. Die Daten werden in sogenannten Model Klassen verwaltet
und von View Klassen angezeigt, wobei eine 1:n Beziehung besteht. Die
Kommunikation zwischen den Klassen erfolgt anonym �ber
Benachrichtigungen (siehe auch \ref{Benachrichtigungen}). 

Eine Model Klasse stellt �ber get-Methoden oder Iteratoren den Zustand
zur Verf�gung. Dieser Zustand wird von einer View-Klasse dem
Anwendungszweck entsprechend angezeigt. Konkret werden die Bl�cke,
Pins und Verbindungen in POA jeweils durch Datenobjekte
repr�sentiert. Diese werden mit Hilfe von entsprechenden Viewobjekten von der
GridCanvas Klasse angezeigt. 

\subsection{Benachrichtigung}

Die QT Bibliothek erweitert den Sprachumfang von C++ �ber einen
speziellen Kompiler, den Meta-Object-Compiler (MOC), mit einem
Benachrichtigungskonzept. Mit Hilfe von
Benachrichtigungen k�nnen Objekte anderen Objekten Mitteilungen �ber
Zustands�nderungen schicken. Die nahtlose Integration in die
Programmiersprache erm�glicht eine besonders einfache Verwendung des
Mechanismus.

Die Benachrichtigungen (Signals) werden durch das Schl�sselwort emit
gesendet und von speziellen Methoden (Slots) empfangen. Die Signals
und Slots werden in den Klassendeklarationen angegeben. Die Verbindung
zwischen zwei Objekten wird mittels der connect Methode hergestellt. 

In POA wird an meheren Stellen gebrauch von dem Signal-Slot
Mechanismus gemacht.

\subsection{Singleton}

Das Singleton Muster wird verwendet, wenn ein Objekt nur genau einmal
instanziert werden soll. Dadurch lassen sich auch in
objekt-orientierten Systemen, �quivalent zu globalen Variablen in
imperativen Systemen, global verf�gbare Objekte realisieren,
die �ber eine definierte Schnittstelle angesprochen werden k�nnen. 

In POA wird nur an e


\section{Module}

Die Architektur von POA l�sst sich gut in stark zusammenh�ngende
Module aufteilen, die eine gerine Kopplung aufweisen. In den folgenden
Abschnitten sind diese Module im Detail beschrieben.

Die Zuordnung von einzelnen Klassen zu Modulen l�sst sich in der Regel
aus dem Klassennamen ableiten oder ist aus dem POA
Software-Architektur anhand der Schattierung ersichtlich.

\subsection{Datenmodell (Model)}

Die Klassen in dem 

\subsection{View}

\subsection{Scheduling}

Das Scheduling Modul 

\subsection{Download}

\subsection{Routing}

\subsection{Problemreport}

\subsection{Querschnittsmodule}



\section{Persistenz}

\subsection{XML Format}




%%% Local Variables: 
%%% TeX-master: "angebot"
%%% End: 
%%% vim:tw=79:
