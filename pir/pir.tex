% Fachstudie NEXUS "Personen im Raum"
% $Id: pir.tex,v 1.3 2005/02/02 13:54:44 garbeam Exp $

\documentclass{ieee}

%------------------------------------------------------------------------- 
% 
% Use \documentclass[pagenumbers]{ieee}
%
% to produce page numbers, bottom centered, in the output. Default is 
% no page numbers for camera-ready copy.
%
%------------------------------------------------------------------------- 

\usepackage{times}

\begin{document}

\title{Fachstudie 'Personen im Raum'}
\author{Steffen Keul, Marcel Kilgus, Anselm Garbe\\
Universit�t Stuttgart, Institut f�r Informatik, Nexus Forschungsprojekt}

\maketitle
\thispagestyle{empty}

\begin{abstract}
In dieser Studie werden Algorithmen zur Bilderkennung von Personen im
Raum anhand von Receiver Operator Characteristic Curves
gegen�bergestellt und abschliessend eine Empfehlung ausgerpochen.
\end{abstract}



%------------------------------------------------------------------------- 
\section{Einleitung}
% TODO: Hintergrund etc.


%------------------------------------------------------------------------- 
\section{Szenarien}

Personen

- keine Person
- eine Person
- Gruppe von Personen (verteilt im Raum, dichte Gruppe (Kn�ul))

Beleuchtung

- Tageslicht (Mittag), ohne Raumbeleuchtung
- D�mmerung
- Elektrische in der Nacht
- ohne Licht

Kameroptionen

- Beleuchgtungselimierung

- hohem Kontrast (Farbenfroh)
- normal
- niedrigen Kontrast (grau)

Algorithmen-Parametrisierung



%------------------------------------------------------------------------- 
\section{Algorithmen}


%------------------------------------------------------------------------- 
\section{Receiver Operator Characteristic Curves}

\begin{figure}[h]
   \caption{Example of caption.}
\end{figure}

%\noindent
Long captions should be set as in 
\begin{figure}[h] 
   \caption{Example of long caption requiring more than one line. It is 
     not typed centered but aligned on both sides and indented with an 
     additional margin on both sides of 1~pica.}
\end{figure}

%------------------------------------------------------------------------- 
\section{Fazit}

%------------------------------------------------------------------------- 
\nocite{ex1,ex2}
\bibliographystyle{ieee}
\bibliography{ieee}

\end{document}

