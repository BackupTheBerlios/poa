%%%%%%%%%%%%%%%%%%%%%%%%%%%%%%%%%%%%%%%%%%%%%%%%%%%%%%%%%%%%%%%%%%%%%%%%%%%%%%%
%% StuPro B, "Programmierumgebung Offener Antrieb" (POA)
%% Anforderungen
%% $Id: lieferumfang.tex,v 1.1 2003/06/01 15:35:51 squig Exp $
%% Achtung: Diese Datei wird ins Angebot inkludiert!
%%%%%%%%%%%%%%%%%%%%%%%%%%%%%%%%%%%%%%%%%%%%%%%%%%%%%%%%%%%%%%%%%%%%%%%%%%%%%%%

%%%%%%%%%%%%%%%%%%%%%%%%%%%%%%%%%%%%%%%%%%%%%%%%%%%%%%%%%%%%%%%%%%%%%%%%%%%%%%%
%% Lieferumfang
\chapter{Lieferumfang}
Nachfolgend werden die auszuliefernden Dokumente und Softwareprogramme
aufgef�hrt. Die Sprache der Dokumente und der Benutzungsoberfl�che ist
Deutsch. Die Kommentierung des Quellcodes erfolgt in Englisch.

\begin{itemize}
\item Benutzerhandbuch in Form einer Online-Hilfe
\item Entwurfsdokument, das die Packete und Klassen mit deren
  Beziehungen darstellt
\item Software
  \begin{itemize}
  \item Ausf�hrbares Programm
  \item README Datei
  \item Kommentierter Quellcode
  \end{itemize}
\end{itemize}

\section{Andere Resultate}
Nicht zum Lieferumfang geh�ren alle sonstigen Dokumente, die w�hrend
des Projekts entstehen, z.B. die Reviewprotokolle.

\section{Externe Bibliotheken}
Bibliotheken, die f�r die Ausf�hrung der ausgelieferten Software
erforderlich sind werden in dem Dokument README aufgef�hrt, geh�ren
aber nicht zum Lieferumfang.

\section{Lizensierung}
Die erstellten Dokumente und die erstellte Software werden unter der
General Public License (GPL, siehe
\url{http://www.gnu.org/licenses/gpl.txt}) lizensiert und auf
http://poa.berlios.de zur Verf�gung gestellt. Desweiteren wird dem
Auftraggeber eine uneingeschr�nkte Lizens f�r Zwecke der Forschung und
Lehre einger�umt.

Insbesondere wird keine Gew�hrleistung �bernommen die �ber die in
diesem Angebot vereinbarte hinaus geht. Die Urheberrechte verbleiben
bei den Autoren. Der Auftraggeber kann die Software und die Dokumente
zu den im Rahmen der Lizenz vereinbarten Bedingungen modifizieren und
nutzen.

%%% Local Variables: 
%%% TeX-master: "angebot"
%%% End: 
