%%%%%%%%%%%%%%%%%%%%%%%%%%%%%%%%%%%%%%%%%%%%%%%%%%%%%%%%%%%%%%%%%%%%%%%%%%%%%%%
%% StuPro B, "Programmierumgebung Offener Antrieb" (POA)
%% Angebot
%% $Id: begriffslexikon.tex,v 1.4 2003/06/12 19:27:53 garbeam Exp $
%%%%%%%%%%%%%%%%%%%%%%%%%%%%%%%%%%%%%%%%%%%%%%%%%%%%%%%%%%%%%%%%%%%%%%%%%%%%%%%
\newcommand{\begriff}[2]
{\item \bfseries{#1} \textnormal{#2}}

\chapter{Begriffslexikon}
\begin{itemize}

\begriff{Block}{siehe Funktionsblock}

\begriff{Bus}{Mehrere Signalleitungen, die von einem Funktionsblock ausgehen.}

\begriff{Core}{Eine festgelegte Logik mit bekannter Laufzeit.}

\begriff{CPLD}{Complex Programmable Logic Device}

\begriff{CPU}{Ein CPU-Block (auf Basis von NIOS).}

\begriff{Download}{Herunterladen eines vollst�ndig compilierten CPLD-Layouts
auf den CPLD.}

\begriff{Funktionsblock}{Virtuelles Abbild eines Cores, einer CPU oder
eines Ein- bzw. Ausgabeblocks.}

\begriff{Plausibilit�tspr�fung}{Pr�fung des CPLD-Layouts auf
un- und falsch verkn�pfte E/As.}

\begriff{Optimierung}{Verk�rzung der Gesamtlaufzeit eines CPLD-Takt-Zyklus.}

\end{itemize}

%%% Local Variables: 
%%% TeX-master: "angebot"
%%% End: 
%%% vim:tw=79:
