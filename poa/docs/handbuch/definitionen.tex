%%%%%%%%%%%%%%%%%%%%%%%%%%%%%%%%%%%%%%%%%%%%%%%%%%%%%%%%%%%%%%%%%%%%%%%%%%%%%%%
%% StuPro B, "Programmierumgebung Offener Antrieb" (POA)
%% Spezifikation
%% $Id: definitionen.tex,v 1.1 2004/03/03 17:25:34 neco Exp $
%% Achtung: Diese Datei wird in das Handbuch inkludiert!
%%%%%%%%%%%%%%%%%%%%%%%%%%%%%%%%%%%%%%%%%%%%%%%%%%%%%%%%%%%%%%%%%%%%%%%%%%%%%%%
\newcommand{\begriff}[2]
{\item \bfseries{#1} \textnormal{#2}}

\chapter {Definitionen}
\begin{itemize}

\begriff{Ausgang}{Eine Speicheradresse auf die schreibend zugegriffen
  wird.}

\begriff{Block}{siehe Funktionsblock}

\begriff{Bus}{Mehrere Signalleitungen k"onnen zu einem Bus zusammen
  gefasst werden.}

\begriff{CPLD-Layout}{Anordnung von Funktionsbl"ocken und Signalleitungen.}

\begriff{CPLD}{Complex Programmable Logic Device}

\begriff{CPU}{Ein CPU-Block (auf Basis von NIOS).}

\begriff{Core}{Eine festgelegte Logik mit bekannter Laufzeit.}

\begriff{Download}{Herunterladen eines vollst"andig compilierten CPLD-Layouts
auf das CPLD.}

\begriff{Eingang}{Eine Speicheradresse auf die lesend zugegriffen wird.}

\begriff{Episodisches Signal}{Ein Eingang oder Ausgang.}

\begriff{Funktionsblock}{Virtuelles Abbild eines Cores, einer CPU oder
eines Ein- bzw. Ausgabeblocks.}

\begriff{ISWOS}{Minimal-Betriebssystem, dass nach der Initialisierung
  des CPLDs auf den CPUs ausgef"uhrt wird.}

\begriff{Laufzeitermittlung}{Absch"atzung der maximalen Laufzeit eines
  Programms in Milli-Sekunden.}

\begriff{Offset}{Anzahl Takte bevor ein Funktionsblock zum ersten Mal
  angestossen wird.}

\begriff{Optimierung}{Verk"urzung der Gesamtlaufzeit eines CPLD-Zyklus.}

\begriff{Periode}{Anzahl Takte nach denen ein Funktionsblock erneut
  gestartet wird}

\begriff{Plausibilit"aspr"ufung}{Pr"ufung des CPLD-Layouts auf
  unverkn"upfte und falsch verkn"upfte Eingangs/Ausgangspins.}

\begriff{Scheduler}{Hardware, die Funktionsbl"ocke periodisch startet}

\begriff{Signalleitung}{Eine Verbindung zwischen zwei
  Funktionsbl"ocken. Eine Signalleitung hat immer 32-bit.}

\begriff{Takt}{Kleinste Zeiteinheit des Schedulers}

\begriff{Verbindung}{Eine Kopplung von einem Ausgang zu ein oder
  mehreren Eing"angen.}

\end{itemize}

