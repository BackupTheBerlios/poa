%%%%%%%%%%%%%%%%%%%%%%%%%%%%%%%%%%%%%%%%%%%%%%%%%%%%%%%%%%%%%%%%%%%%%%%%%%%%%%%
%% StuPro B, "Programmierumgebung Offener Antrieb" (POA)
%% Projektplan
%% $Id: projektplan.tex,v 1.7 2003/06/14 16:50:34 garbeam Exp $
%% Achtung: Diese Datei wird ins Angebot inkludiert!
%%%%%%%%%%%%%%%%%%%%%%%%%%%%%%%%%%%%%%%%%%%%%%%%%%%%%%%%%%%%%%%%%%%%%%%%%%%%%%%

\newcommand{\milestone}[3]{
\paragraph* {#1}
\vspace{0.3cm}
Ergebnis: \begin {bfseries} #2 \end {bfseries}
\\[0.3cm]
#3
}

\chapter {Projektplan}

\section {Zeitplan}
W�hrend der erste Phase des Projekts wird die grundlegende Architektur
der Software festgelegt und ein Prototyp erstellt. In der zweiten
Phase wird der Prototyp in drei Sprints erweitert.

Zu Anfang jedes Sprints werden in einer priorisierten Liste die
Ziele f�r den Sprint festgelegt und innerhalb von 4 Wochen fertig
gestellt. Am Ende jedes Sprints wird dem Auftraggeber die erstellte
Software zur Pr�fung vorgelegt. 

\begin {center}
  \begin {tabular}[l]{rlrll}
    \multicolumn{1}{c}{\bf ID} & \multicolumn{1}{c}{\bf Bezeichnung} &
    \multicolumn{1}{c}{\bf Aufwand} & \multicolumn{1}{c}{\bf Termin} &
    \multicolumn{1}{c}{\bf Meilenstein} \\
    \multicolumn{2}{c}{} &
    \multicolumn{1}{c}{(in Mh)} \\
    \hline
     1 & Analyse und Angebot & 200 & KW 25 & M1: Angebot \\
     2 & Spezifikation & 300 & & \\
     2 & Entwurf & 300 & & \\
     2 & Prototyp Messe & 300 & 1.10.2003 & M2: Prototyp Messe \\
     2 & Prototyp Erweitert & 300 & 10.11.2003 & M3: Prototyp Erweitert \\
    16 & Sprint 1 & & KW 48 - 51 & \\
    16 & Sprint 2 & & KW 2 - 6 & \\
    16 & Sprint 3 & & KW 8 - 12 & \\
    26 & Auslieferung und Pr�sentation & & 1.4.2004 & M4: Auslieferung\\
    \hline
       & & 1400\\
  \end {tabular}
\end {center}

\section {Kostensch�tzung}

F�r die erste Phase des Projekts wird ein Zeitaufwand von 1400 Stunden
veranschlagt, das entspricht Kosten von 140.000 Euro. Die Bezahlung
der Sprints wird zu Beginn jedes Sprints festgelegt und anhand der
erreichten Ergebnisse vorgenommen.

\section {Meilensteine}
\label {Meilensteine}

Das Projekt ist, wie aus dem Zeitplan ersichtlich, in mehrere Phasen
eingeteilt. Die Meilensteine dienen dazu, den erfolgreichen Abschluss
einer Phase zu �berpr�fen. Zum erfolgreichen Erreichen eines
Meilensteines m�ssen bestimmte Dokumente oder Softwareprogramme
fertiggestellt werden. Diese sind bei den Meilensteinen unter Ergebnis
angegeben. Die Meilensteinabnahme ist erforderlich, um zur n�chsten 
Phase �berzugehen bzw. eine auf den Ergebnissen aufbauende T�tigkeit 
durchzuf�hren. Dies wird nicht getan, bis die Meilensteinabnahme erfolgt ist.

Die erstellten Dokumente oder Softwareprogramme werden dem
Auftraggeber vorgelegt. Sollten die erzielten Ergebnisse nicht
den Erwartungen entsprechen, m�ssen diese nachgebessert
werden. Sollten dadurch erhebliche Verz�gerungen im Projekt auftreten,
muss der Zeitplan entsprechend angepasst werden. Der Auftraggeber nimmt
die Meilensteine ab.

\milestone {M1: Angebot}{Angebot}{Das vorliegende Dokument.}

\milestone {M2: Prototyp Messe}{Entwurf, Lauff�higer Prototyp} {Der
  Prototyp, der auf der Messe pr�sentiert werden soll. Die
  Funktionalit�t wird sich auf eine pseudo interaktive grafische
  Oberfl�che beschr�nken. Sollte der Prototyp stark von den
  Erwartungen abweichen, kann das Projekt hier abgebrochen werden.}

\milestone {M3: Prototyp Erweitert}{Lauff�higer Prototyp} {Ein
  mit erweiterter Funktionalit�t ausgestatteter Prototyp, der als
  Grundlage f�r den ersten Sprint dient.}

\milestone {M4: Auslieferung}{Software}{Auslieferung des erstellten
  Systems.}

\section{Risiken}

\paragraph* {Nicht-Erf�llen von Meilensteinen}
\ \\[0.3cm]
Sollte ein Meilenstein nicht erf�llt werden k�nnen, also die
erwarteten Ergebnisse nicht geliefert werden, muss der Projektplan
entsprechend angepasst werden. Sollte das nicht m�glich sein, da keine
ausreichende Pufferzeiten vorhanden sind, m�ssen die Anforderungen in
�bereinkunft mit dem Auftraggeber reduziert werden.

\paragraph* {Ausfall von Team-Mitgliedern}
\ \\[0.3cm]
Sollten Teammitglieder unerwartet ausfallen, m�ssen die Anforderungen
gegebenenfalls reduziert werden und Aufgabenverteilung angepasst
werden. Wenn dies nicht m�glich ist, muss in �bereinkunft mit dem
Auftraggeber eine andere L�sung (z.B. Reduktion der Anforderungen)
gefunden werden.

\paragraph*{Prozessprobleme}
\ \\[0.3cm]
Falls sich herausstellt, da� das gew�hlte Prozessmodell ungeignet ist,
da das Team oder der Auftraggeber nicht mit den Bedingungen des
Modells zurecht kommen, kann ein anderes Modell gew�hlt werden.

%%% Local Variables: 
%%% TeX-master: "angebot"
%%% End: 
%%% vim:tw=79:
