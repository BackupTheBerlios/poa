%%%%%%%%%%%%%%%%%%%%%%%%%%%%%%%%%%%%%%%%%%%%%%%%%%%%%%%%%%%%%%%%%%%%%%%%%%%%%%%
%% StuPro B, "Programmierumgebung Offener Antrieb" (POA)
%% Entwurf
%% $Id: einleitung.tex,v 1.2 2004/02/11 14:29:18 squig Exp $
%% Achtung: Diese Datei wird in den Entwurf inkludiert!
%%%%%%%%%%%%%%%%%%%%%%%%%%%%%%%%%%%%%%%%%%%%%%%%%%%%%%%%%%%%%%%%%%%%%%%%%%%%%%%

\chapter {Einleitung}

Dieses Dokument beschreibt die Entwurfsstrategien von POA. Es richtet
sich an Entwickler mit Kenntnissen in C++ und Entwurfsmuster, die die
Architektur von POA verstehen m�chten. Das Dokument soll die
Einarbeitung erleichtern und sicherstellen, dass die bestehenden
Strukturen bei zuk�nftigen Erweiterung erhalten bleiben.

Eine detailierte Beschreibung der einzelnen Funktionen ist nicht Teil
dieses Dokuments. Diese befindet sich in Form von JavaDoc Kommentaren,
die mit Doxygen ausgewertet werden k�nnen in den Header Dateien.

Der POA Entwurf wurde nach folgenden Kritieren optimiert:

\begin{itemize}
\item Leichte Erweiterbarkeit
\item Hohe �nderbarkeit
\end{itemize}

Erreicht wurden diese Eigenschafen in erster Linie durch ein einfaches
Design. Es wurden nur die notwendigen Klassen implementiert, keine
Technologie auf Vorrat angelegt. 

%%% Local Variables: 
%%% TeX-master: "entwurf"
%%% End: 
%%% vim:tw=79:
