% Fachstudie NEXUS "Personen im Raum"
% $Id: pir.tex,v 1.9 2005/03/13 13:32:59 garbeam Exp $

\documentclass{article}
\usepackage[german]{babel} 
\usepackage[latin1]{inputenc}
\usepackage[T1]{fontenc}
\usepackage{fancyhdr}
\usepackage{times}
\usepackage{url}

\begin{document}

\title{Fachstudie 'Personen im Raum'}
\author{Steffen Keul, Marcel Kilgus, Anselm Garbe\\
\small Universit�t Stuttgart, Institut f�r Informatik, Nexus Forschungsprojekt}

\maketitle
\thispagestyle{empty}

\begin{abstract}
In dieser Studie werden Algorithmen zur Bilderkennung von Personen im
Raum anhand von Receiver Operator Characteristic Curves
gegen�bergestellt und abschliessend eine Empfehlung ausgesprochen.
\end{abstract}

%------------------------------------------------------------------------- 
\section{Einleitung}
Ziel dieser Fachstudie ist es, verschiedene Bilderkennungs-Algorithmen
hinsichtlich ihrer Eignung zur Erkennung von Bewegungen zu bewerten. Ein
Algorithmus, der zuverl�ssig Bewegungen auf Bild-Sequenzen erkennt,
kann dabei f�r Aussagen zur Erkennung von Objekten, wie z.B. Personen,
in Betracht gezogen werden (Tracking).
Personen sind in der Regel keine starren Objekte und zeigen Ver�nderungen
in Form von Bewegungen zwischen einzelnen Bildern einer Bild-Sequenz auf.
Die Erkennung solcher Bewegungen kann zur Existenzaussage von Personen im
Bildausschnitt verwendet werden. Dar�ber hinaus wird untersucht, in wie
weit solche Existenzaussagen zur Z�hlung von Personen herangezogen werden
k�nnen.
Neben diesen Bewertungen wird auch ein kleiner Vergleich auf vorhandene
technische Systeme zur Bewegungserkennung gegeben, die auf Bilderkennung
basieren.


%------------------------------------------------------------------------- 
\section{Szenarien}

Untersucht wird die Bilderkennung von einer Person im Raum bei drei
unterschiedlichen Bidlkontrasteinstellungen der verwendeten
Kamera, Modell MOBOTIX M1, siehe auch \url{http://www.mobotix.com}.

\begin{figure}[h]
    \begin{center}
        \begin{tabular}{l|l}
        \textbf{Format:} & JPEG\\ \hline
        \textbf{Kompression:} & Standard\\ \hline
        \textbf{Aufl�sung (Pixel):} & 640x480\\ \hline
        \textbf{Automatischer Lichtfilter:} & aus\\ \hline
        \textbf{Rauschenunterdr�ckung:} & aus\\ \hline
        \textbf{Bildsch�rfe (-2..20):} & 0\\
        \end{tabular}
    {\small \caption{Feste Bildeigenschaften}}
    \end{center}
\end{figure}

\subsection{Bildkontrasteinstellungen}
Die drei im Szenario verwendeten Bildkontrasteinstellungen k�nnen grob
in {\it hoch}, {\it normal} und {\it niedrig} eingestuft werden.
Motivation f�r diese Wahl war die Vermutung, dass ein geringes
Farbspektrum (niedriger Kontrast)
bessere Erkennungsraten liefert, als ein hohes Farbspektrum.

\begin{figure}[h!]
    \begin{center}
        \begin{tabular}{l|l}
        \textbf{Helligkeit (-10..10):} & 5\\ \hline
        \textbf{S�ttigung (-10..10):} & 5\\
        \end{tabular}
    {\small \caption{Hoher Bildkontrast}}
    \end{center}
\end{figure}

\begin{figure}[h!]
    \begin{center}
        \begin{tabular}{l|l}
        \textbf{Helligkeit (-10..10):} & 0\\ \hline
        \textbf{S�ttigung (-10..10):} & 0\\
        \end{tabular}
    {\small \caption{Hoher Bildkontrast}}
    \end{center}
\end{figure}

\begin{figure}[h!]
    \begin{center}
        \begin{tabular}{l|l}
        \textbf{Helligkeit (-10..10):} & -5\\ \hline
        \textbf{S�ttigung (-10..10):} & -5\\
        \end{tabular}
    {\small \caption{Hoher Bildkontrast}}
    \end{center}
\end{figure}


\subsection{Offene Fragen}

Die Bilderkennung von mehreren Personen im Raum sowie von keiner Person
wurde bis auf einzelne Feldversuche nicht genauer untersucht.
Anzunehmen ist, dass die Erkennungsrate von mehreren Personen im Raum
etwas h�her liegt, als bei einer einzelnen Person im Raum.
Die Erkennungsrate von Personen im Raum, obwohl sich keine Person im
Raum aufh�lt, wird als sehr gering angenommen.

Die Entwickler von OpenCV bieten keine Beispiele f�r optische
Bewegungsflusserkennung an, so dass der Schluss naheliegt, dass die
vorhandenen Algorithmen entweder ungeeignet sind oder nie getestet
wurden.

%------------------------------------------------------------------------- 
\section{Algorithmen}

Mit Ausnahme eines selbst implementierten Algorithmus, sind alle anderen
Algorithmen in der OpenCV Bibliothek\cite{opencv} von Intel enthalten.
Zur Erkennung von Personen werden Algorithmen untersucht, die Bewegungen
zwischen zwei Bildern erkennen k�nnen. Signifikante Bewegungen gelten
als Bewegung von gr��eren Objekten, darunter sind Bewegungen von
Menschen inbegriffen.

\subsection{CalcOpticalFlowHS}
Dieser Algorithmus berechnet den optischen Bewegungsfluss f�r alle Pixel
zwischen zwei Bildern nach dem Horn \& Schunk \cite{Horn81} Verfahren.

\subsection{CalcOpticalFlowLK}
Dieser Algorithmus berechnet den optischen Bewegungsfluss f�r alle Pixel
zwischen zwei Bildern nach dem Lucas \& Kanade \cite{Lucas81} Verfahren.

\subsection{CalcOpticalFlowBM}
Dieser Algorithmus berechnet den optischen Bewegungsfluss f�r
Segmente einer frei definierbaren Gr��e zwischen zwei Bildern.
Um so gr�ber die Segmente gew�hlt werden, desto performanter arbeitet der
Algorithmus.

\subsection{CalcOpticalFlowPyrLK}
Dieser Algorithmus \cite{Bouget00} basiert auf einer iterativen Version der optischen
Fluss-Pyramiden nach Lucas \& Kanade. Er berechnet den optischen
Bewegungsfluss auf einer Menge von Punkten zwischen zwei Bildern. Es
werden Bewegungen mit Sub-Pixel Genauigkeit gefunden.

\subsection{SegmentMotion}
Dieser Algorithmus sucht alle bewegten Segmente zwischen zwei Bildern
und vergibt f�r jede Bewegung in einer Segmentmaske Werte zwischen
$1..N$, je nach Bewegungsintensit�t.

\subsection{Eigene Implementierung}

%------------------------------------------------------------------------- 
\section{Receiver Operator Characteristic Curves}

%------------------------------------------------------------------------- 
\section{Empfehlung}
Im Rahmen dieser Fachstudie empfehlen wir die optische Flusserkennung
nach \cite{KIL05}, die auch im Source Code im Anhang vorliegt.

\begin{thebibliography}{99}
\bibitem{KIL05} Marcel Kilgus, Optische Flusserkennung mit 15 LOC, 2005.
\bibitem{opencv} http://opencvlibrary.sf.net
\bibitem{Horn81} Berthold K. P. Horn und Brian G. Schunck. Determining
Optical Flow. Artificial Intelligence, 17, pp. 185-203, 1981.
\bibitem{Lucas81} Lucas, B. und Kanade, T. An Iterative Image
Registration Technique with an Application to Stereo Vision, Proc. of
7th International Joint Conference on Artificial Intelligence (IJCAI),
pp. 674-679.
\bibitem{Bouget00} Jean-Yves Bouguet. Pyramidal Implementation of the
Lucas Kanade Feature Tracker.
\end{thebibliography}

\end{document}
