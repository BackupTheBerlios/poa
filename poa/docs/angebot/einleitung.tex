%%%%%%%%%%%%%%%%%%%%%%%%%%%%%%%%%%%%%%%%%%%%%%%%%%%%%%%%%%%%%%%%%%%%%%%%%%%%%%%
%% StuPro B, "Programmierumgebung Offener Antrieb" (POA)
%% Projektplan
%% $Id: einleitung.tex,v 1.4 2003/06/14 16:49:17 garbeam Exp $
%% Achtung: Diese Datei wird ins Angebot inkludiert!
%%%%%%%%%%%%%%%%%%%%%%%%%%%%%%%%%%%%%%%%%%%%%%%%%%%%%%%%%%%%%%%%%%%%%%%%%%%%%%%

\chapter {Einleitung}
\section {Beschreibung des Projektplan}

Dieses Dokument ist Teil eines Angebotes �ber eine Projektdurchf�hrung
im Rahmen des Studienprojektes B ``Programmierumgebung Offener
Antrieb'' am ISW, Universit�t Stuttgart.

Der Projektplan gibt zu diesem Zeitpunkt eine grobe Einsch�tzung des
zu erwartenden Projektverlaufes wider. Der Plan wird w�hrend des
Projekts laufend angepasst.

\section {Entwicklungsphilosophie}

Das Projekt ist im Wesentlichen durch zwei Bedingungen gepr�gt:
\begin{itemize}
\item
Auf der funktionalen Ebene sind die Anforderungen unscharf und
unterliegen st�ndigen �nderungen, da die Hardware, die von der zu
erstellenden Software programmiert werden soll, parallel dazu
entwickelt wird. Daraus ergeben sich auch w�hrend des Projektverlaufes 
neue Anforderungen.
\item
Die zweite Einschr�nkung betrifft den zeitlichen Ablauf des Projektes.
Bis Mitte Oktober 2003 wird f�r eine Pr�sentation ein lauff�higer
Prototyp der Software ben�tigt.
\end{itemize}
Um diesen Bedingungen gerecht zu werden dient ein
Inkrementell-Iteratives Prozessmodell\footnote{siehe Balzert, Helmut: Lehrbuch
der Software-Technik, Band 2, Spektrum, 1998} zur Entwicklung. Im
Wesentlichen wird zu Anfang des Projektes eine Anforderungsanalyse
durchgef�hrt und ein Prototyp erstellt. Dieser wird dann in
sogenannten Sprints verbessert bis der gew�nschte Zustand erreicht
ist.

%%% Local Variables: 
%%% TeX-master: "angebot"
%%% End: 
%%% vim:tw=79:
