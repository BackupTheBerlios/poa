% Fachstudie NEXUS "Personen im Raum"
% $Id: pir.tex,v 1.4 2005/02/09 14:19:49 garbeam Exp $

\documentclass{article}
\usepackage[german]{babel} 
\usepackage[latin1]{inputenc}
\usepackage[T1]{fontenc}
\usepackage{fancyhdr}
\usepackage{times}
\usepackage{url}

\begin{document}

\title{Fachstudie 'Personen im Raum'}
\author{Steffen Keul, Marcel Kilgus, Anselm Garbe\\
\small Universit�t Stuttgart, Institut f�r Informatik, Nexus Forschungsprojekt}

\maketitle
\thispagestyle{empty}

\begin{abstract}
In dieser Studie werden Algorithmen zur Bilderkennung von Personen im
Raum anhand von Receiver Operator Characteristic Curves
gegen�bergestellt und abschliessend eine Empfehlung ausgesprochen.
\end{abstract}

%------------------------------------------------------------------------- 
\section{Einleitung}
% TODO: Hintergrund etc.


%------------------------------------------------------------------------- 
\section{Szenarien}

Untersucht wird die Bilderkennung von einer Person im Raum bei drei
unterschiedlichen Bidlkontrasteinstellungen der verwendeten
Kamera, Modell MOBOTIX M1, siehe auch \url{http://www.mobotix.com}.

\begin{figure}[h]
    \begin{center}
        \begin{tabular}{l|l}
        \textbf{Format:} & JPEG\\ \hline
        \textbf{Kompression:} & Standard\\ \hline
        \textbf{Aufl�sung (Pixel):} & 640x480\\ \hline
        \textbf{Automatischer Lichtfilter:} & aus\\ \hline
        \textbf{Rauschenunterdr�ckung:} & aus\\ \hline
        \textbf{Bildsch�rfe (-2..20):} & 0\\
        \end{tabular}
    {\small \caption{Feste Bildeigenschaften}}
    \end{center}
\end{figure}

\subsection{Bildkontrasteinstellungen}
Die drei im Szenario verwendeten Bildkontrasteinstellungen k�nnen grob
in {\it hoch}, {\it normal} und {\it niedrig} eingestuft werden.
Motivation f�r diese Wahl war die Vermutung, dass ein geringes
Farbspektrum (niedriger Kontrast)
bessere Erkennungsraten liefert, als ein hohes Farbspektrum.

\begin{figure}[h!]
    \begin{center}
        \begin{tabular}{l|l}
        \textbf{Helligkeit (-10..10):} & 5\\ \hline
        \textbf{S�ttigung (-10..10):} & 5\\
        \end{tabular}
    {\small \caption{Hoher Bildkontrast}}
    \end{center}
\end{figure}

\begin{figure}[h!]
    \begin{center}
        \begin{tabular}{l|l}
        \textbf{Helligkeit (-10..10):} & 0\\ \hline
        \textbf{S�ttigung (-10..10):} & 0\\
        \end{tabular}
    {\small \caption{Hoher Bildkontrast}}
    \end{center}
\end{figure}

\begin{figure}[h!]
    \begin{center}
        \begin{tabular}{l|l}
        \textbf{Helligkeit (-10..10):} & -5\\ \hline
        \textbf{S�ttigung (-10..10):} & -5\\
        \end{tabular}
    {\small \caption{Hoher Bildkontrast}}
    \end{center}
\end{figure}


\subsection{Offene Fragen}

Die Bilderkennung von mehreren Personen im Raum sowie von keiner Person
wurde bis auf einzelne Feldversuche nicht genauer untersucht.
Anzunehmen ist, dass die Erkennungsrate von mehreren Personen im Raum
etwas h�her liegt, als bei einer einzelnen Person im Raum.
Die Erkennungsrate von Personen im Raum, obwohl sich keine Person im
Raum aufh�lt, wird als sehr gering angenommen.

%------------------------------------------------------------------------- 
\section{Algorithmen}

Alle untersuchten Algorithmen basieren auf oder sind in der OpenCV \cite{opencv}
Bibliothek der Intel Corp. enthalten.




%------------------------------------------------------------------------- 
\section{Receiver Operator Characteristic Curves}

%------------------------------------------------------------------------- 
\section{Empfehlung}
Im Rahmen dieser Fachstudie empfehlen wir die optische Flusserkennung
nach \cite{KIL05}, die auch im Source Code im Anhang vorliegt.

\begin{thebibliography}{99}
\bibitem{KIL05} Marcel Kilgus, Optische Flusserkennung mit 15 LOC, 2005.
\bibitem{opencv} http://opencv.sf.net
\end{thebibliography}

\end{document}
