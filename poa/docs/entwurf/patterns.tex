%%%%%%%%%%%%%%%%%%%%%%%%%%%%%%%%%%%%%%%%%%%%%%%%%%%%%%%%%%%%%%%%%%%%%%%%%%%%%%%
%% StuPro B, "Programmierumgebung Offener Antrieb" (POA)
%% Entwurf
%% $Id: patterns.tex,v 1.1 2004/06/16 11:55:21 squig Exp $
%% Achtung: Diese Datei wird in den Entwurf inkludiert!
%%%%%%%%%%%%%%%%%%%%%%%%%%%%%%%%%%%%%%%%%%%%%%%%%%%%%%%%%%%%%%%%%%%%%%%%%%%%%%%

\chapter{Entwurfsmuster}

Der Entwurf von POA integriert bekannte und weitverbreitete
Entwurfsmuster, um die Struktur des Entwurfs leicht verst�ndlich zu
machen und f�r zuk�nftige Erweiterungen im gr��tm�glichen Ma�e offen
zu sein.

Entwufsmuster beschreiben eine L�sungsstrategie f�r ein abstraktes
Problem. Die Struktur des Codes wird so weitgehend standardisiert und
erm�glicht einem Wartungsingenieur eine schnelle
Einarbeitung. Weiterhin bieten sie Richtlinien an, wie weitere
Funktionalit�t in das System integriert werden kann.

\section{Model-View-Controller}
\label{model_view_controller}

In dem Model-View-Controller Entwurfsmuster wird die Verarbeitung der
Daten vollst�ndig von der Visualisierung unabh�ngig gehalten. Die
Daten werden in sogenannten Model Klassen verwaltet und von View
Klassen angezeigt, wobei eine 1:n Beziehung besteht, das hei�t mehrere
Views k�nnen die selben Daten darstellen. Es werden auch verschiedene
Ansichten auf die selben Daten erm�glicht. Die Kommunikation von
Model- zu View-Klassen erfolgt anonym �ber Benachrichtigungen (siehe
auch \ref{Benachrichtigungen}).

Eine Model Klasse stellt �ber get-Methoden oder Iteratoren den Zustand
zur Verf�gung. Dieser Zustand wird von einer View-Klasse dem
Anwendungszweck entsprechend angezeigt. Konkret werden die Bl�cke,
Pins und Verbindungen in POA jeweils durch Datenobjekte
repr�sentiert. Diese werden mit Hilfe von entsprechenden Viewobjekten von der
GridCanvas Klasse angezeigt. 

\section{Observer}

Die QT Bibliothek erweitert den Sprachumfang von C++ �ber einen
speziellen Compiler, den Meta-Object-Compiler (MOC), mit einem
Benachrichtigungskonzept. Mit Hilfe von
Benachrichtigungen k�nnen Objekte anderen Objekten Mitteilungen �ber
Zustands�nderungen schicken. Die nahtlose Integration in die
Programmiersprache erm�glicht eine besonders einfache Verwendung des
Mechanismus.

\label{Benachrichtigungen}
Die Benachrichtigungen (Signals) werden durch das Schl�sselwort emit
gesendet und von speziellen Methoden (Slots) empfangen. Die Signals
und Slots werden in den Klassendeklarationen angegeben. Die Verbindung
zwischen zwei Objekten wird mittels der connect Methode hergestellt. 

In POA wird an mehreren Stellen Gebrauch von dem Signal-Slot
Mechanismus gemacht: 
\begin{enumerate}
\item Wenn sich Einstellungen ver�ndern 
\item Wenn ein Projekt ver�ndert wurde
\item Um die Kommunikation zwischen Pins und den zugeh�rigen Bl�cken
und Konnektoren zu erm�glichen.
\item Um Ver�nderungen zwischen Models und Views zu synchronisieren.
\end{enumerate}

\section{Strategie}

Das Strategiemuster erm�glicht, den Algorithmus f�r eine Probleml�sung
austauschbar zu machen. 

In POA wurde die Routing Schnittstelle f�r die Verbindungen zwischen
den Bl�cken abstrakt definiert und konkret mit zwei verschiedenen
Algorithmen implementiert. Die Algorithmen haben unterschiedliche
Eigenschaften in Bezug auf die Laufzeit und Qualit�t des Ergebnisses. 

\section{Singleton}

Das Singleton Muster wird verwendet, wenn eine Klasse nur genau einmal
instanziert werden soll. Dadurch lassen sich auch in
objekt-orientierten Systemen, �quivalent zu globalen Variablen in
imperativen Systemen, global verf�gbare Objekte realisieren,
die �ber eine definierte Schnittstelle angesprochen werden k�nnen. 

In POA ist lediglich die Settings Klasse als Singleton implementiert.

%%% Local Variables: 
%%% TeX-master: "entwurf"
%%% End: 
%%% vim:tw=79:
